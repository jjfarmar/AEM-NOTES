\documentclass{article}
\usepackage[utf8]{inputenc}
\usepackage[english]{babel}
\usepackage{lipsum,lineno}
\usepackage{amsmath}
\usepackage{amssymb}
\usepackage{mathtools}
\usepackage{tikz}
\usepackage[dvipsnames]{xcolor}

\newcommand*\circled[1]{\tikz[baseline=(char.base)]{%
            \node[shape=circle,fill=blue!20,draw,inner sep=2pt] (char) {#1};}}

\usepackage{enumitem}

\usepackage[ 
  paperwidth = 168.3mm,
  paperheight = 260.4mm,
  top = 6mm,
  bottom = 7mm,
  outer = 6mm,
  inner = 20mm
]{geometry}

\setlength\parindent{0pt}

\newcommand{\CommentFontSize}{21}
\newcommand{\CommentSkipMult}{25}

\newcommand{\UserFontSize}{16}
\newcommand{\UserSkipMult}{0}

\newcommand{\DateFontSize}{15}
\newcommand{\DateSkipMult}{0}

\newenvironment{commentTextFont}
 {\fontfamily{mdugm}%
  \fontsize{\CommentFontSize}{\CommentSkipMult}%
  \selectfont}
 {\par}

 \newenvironment{userFont}
 {\fontfamily{mdugm}%
  \fontsize{\UserFontSize}{\UserSkipMult}%
  \selectfont}
 {\par}

 \newenvironment{dateFont}
 {\fontfamily{mdugm}%
  \fontsize{\DateFontSize}{\DateSkipMult}%
  \selectfont}
 {\par}

 \newcommand{\uComment}[3]{%
  \filbreak
  \begin{commentTextFont}#1\end{commentTextFont}%
  \vspace*{0.5cm}
  \begin{userFont}\textit{#2}\hspace*{\fill}%
  \begin{dateFont}#3\end{dateFont}\end{userFont}%
  \vspace*{0.8cm}
}

\newcommand{\nobarfrac}{\genfrac{}{}{0pt}{}}

\newcommand{\numberset}[1]{\mathbb{#1}}
\newcommand{\nat}{\numberset{N}}

\DeclarePairedDelimiter\abs{\lvert}{\rvert}%
\DeclarePairedDelimiter\norm{\lVert}{\rVert}%

\makeatletter
\let\oldabs\abs
\def\abs{\@ifstar{\oldabs}{\oldabs*}}
%
\let\oldnorm\norm
\def\norm{\@ifstar{\oldnorm}{\oldnorm*}}
\makeatother

\author{}

\begin{document}

\raggedright



\title{Advanced Engineering Mathematics}

\date{\vspace{-5ex}}

\maketitle

\section{SERIES}
\indent A \underline{sequence} is a list of terms that have been arranged in a certain order.\medskip

\noindent A \underline{series} is the sum of all the terms in a sequence. However, there has to be a definite relationship
between all the terms.
\subsection{ARITHMETIC SEQUENCE}


A \underline{sequence} is \underline{arithmetic} if \({d} \in \mathbb{R} \ni \forall {k} \in \mathbb{Z}^{+} \), 
\[a_{k+1} = a_{k+d}\]
where \(d = a_{k+1} - a_k\) is the common difference

\noindent and \(d = a_k + (n-k)d\) is the nth term of the sequence\medskip


\noindent \underline{NOTATION}:
 $ \{ a_n \} $ or $ \{ a_n \}_{n=1}^{\infty}$ 

\subsubsection{ARITHMETIC SERIES}
Partial Sum:

\[ S_n = \frac{n}{2}(2a_1 + (n-1)d)\]
Or
\[ S_n = \frac{n}{2}(a_1 + a_n)\]

\subsection{GEOMETRIC SEQUENCE}

A \underline{sequence} is \underline{geometric} if \(a_1 \ne 0 \) and if \( {r} \in \mathbb{R} \ne 0 \ni \forall k \in \mathbb{Z}\),

\[a_{k+1} = a_{k} r\]

\noindent where \(r = \frac{a_{k+1}}{a_k}\) is the common ratio\\
and \(a_n = a_k r^{n-k}\) is the nth term

\subsubsection{GEOMETRIC SERIES}

 \[ \sum_{n=1}^{\infty} a^{n-1} = {a + ar + ar^{2} + ...} \]

\noindent is convergent if \(\abs{r} < 1 \)\\
and the sum is 

\[ \sum_{n=1}^{\infty} ar^{n-1} = \frac{a}{1-r},  \abs{r} < 1\]

\subsection{CONVERGENCE}

A series \underline{converges} when the infinite sequence of the partial sums have a finite limit.\medskip

\noindent Any series in which individual terms approach zero \underline{converges.}\medskip

If $\sum_{n=1}^{\infty}a_n$ is \underline{convergent} then \( \lim_{n \to \infty} a_n = 0 \)

\subsection{DIVERGENCE}

A series \underline{diverges} when the infinite sequence of the partial sums does not have a finite limit.\medskip

\noindent Any series in which individual terms does not approach zero \underline{diverges}.

Given a series \( \ \sum_{n=1}^{\infty} a_n = a_1 + a_2 + a_3 + ...\),
\[S_n = \sum_{i = 1}^{n} a_i = a_1 + a_2 + ... + a_n\]

\noindent If $\{S_n\}$ is \underline{convergent} and \(\lim_{n \to \infty} S_n = s\) exists as a real number, then \( \sum a_n\) is called
\underline{convergent} and 

\[a_1 + a_2 + .... + a_n + ... = s\]

Or

\[\sum_{n=1}^{\infty} a_n = s\]

\noindent where s is the sum. Otherwise, the series is \underline{divergent}.

\subsection{TEST FOR DIVERGENCE}

If \( lim_{n \to \infty} a_n\) does not exist or
if \( lim_{n \to \infty} a_n \neq 0\), then the series \( \sum_{n=1}^{\infty} a_n\) is \underline{divergent}.

\subsection{PROPERTIES OF CONVERGENT SERIES}

If \( \sum a_n \) and \( \sum b_n \) are \underline{convergent} series, then so are:

\begin{enumerate}[label=\protect\circled{\roman*}]
  \item \( \sum_{n=1}^{\infty} ca_n = c \sum_{n=1}^{\infty} a_n\)
  \item \( \sum_{n=1}^{\infty} ( a_n + b_n ) = \sum_{n=1}^{\infty} a_n + \sum_{n=1}^{\infty} b_n\)
  \item \( \sum_{n=1}^{\infty} ( a_n - b_n ) = \sum_{n=1}^{\infty} a_n - \sum_{n=1}^{\infty} b_n\)
\end{enumerate}

\subsection{INTEGRAL TEST}

Suppose \(f\) is a \textit{continuous}, \textit{positive}, \textit{decreasing} function on \( [1,\infty)\).\\
\noindent Then \( \sum_{n=1}^{\infty} a_n \) is \underline{convergent} if and only if the improper integral \( \int_{1}^{\infty} f(x) dx \) is
convergent.

\begin{enumerate}[label=\protect\circled{\roman*}]
  \item If \( \int_{1}^{\infty} f(x) dx \) is \underline{convergent}, then \( \sum_{n=1}^{\infty} a_n\) is \underline{convergent}.
  \item If \( \int_{1}^{\infty} f(x) dx \) is \underline{divergent}, then \( \sum_{n=1}^{\infty} a_n\) is \underline{divergent}.
\end{enumerate}

\subsection{\(p\)-SERIES}

The \(p\)-series \( \sum_{n=1}^{\infty} \frac{1}{n^p} \) is \underline{convergent} if \( p > 1\) and \underline{divergent} if \( p \leq 1\).

\subsection{COMPARISON TEST}

Suppose \( \sum a_n \) and \( \sum b_n \) are series with \textit{positive} terms.

\begin{enumerate}[label=\protect\circled{\roman*}]
  \item If \( \sum b_n \) is \underline{convergent} and \( a_n \leq b_n \forall n \), then \( \sum a_n \) is \underline{convergent}. 
  \item If \( \sum b_n \) is \underline{divergent} and \( a_n \geq b_n \forall n \), then \( \sum a_n \) is \underline{divergent}.
\end{enumerate}

\subsection{LIMIT COMPARISON TEST}

Suppose \( \sum a_n\) and \( \sum b_n \) are series with \textit{positive} terms.

If \( \lim_{n \to \infty} \frac{a_n}{b_n} = c\)

where \(c\) is a finite number and \( c > 0 \), then both series either \underline{converge} or \underline{diverge}.\bigskip

\subsection{ALTERNATING SERIES TEST}

If the alternating series

\[ \sum_{n=1}^{\infty} (-1)^{n-1} b_n = b_1 - b_2 + b_3 - b_4 + b_5 - b_6 + ..., b_n > 0 \]

Satisfies 

\begin{enumerate}[label=\protect\circled{\roman*}]
  \item \( b_{n+1} \leq b_n \forall n \)
  \item \( \lim_{n \to \infty} b_n = 0\)
\end{enumerate}

then the series is \underline{convergent}.

\subsection{ABSOLUTE CONVERGENCE}

If \( \sum_{n=0}^{\infty} \abs{a_n}\) \underline{converges}, then \( \sum_{n=0}^{\infty} a_n\) \underline{converges}.

\subsection{CONDITIONAL CONVERGENCE}

If \( \sum a_n \) converges, but \( \abs{a_n} \) does not, \( \sum a_n \) \underline{converges conditionally}.

\subsection{RATIO TEST}

\begin{enumerate}[label=\protect\circled{\roman*}]
  \item If \( \lim_{n \to \infty} \abs{\frac{a_{n+1}}{a_n}} = L < 1\), then \( \sum_{n=1}^{\infty} a_n \) is \underline{absolutely convergent}.
  \item If \( \lim_{n \to \infty} \abs{\frac{a_{n+1}}{a_n}} = L > 1\) or \( \lim_{n \to \infty} \abs{\frac{a_{n+1}}{a_n}} = \infty \), then \( \sum_{n=1}^{\infty} a_n \) is \underline{divergent}.
  \item If \( \lim_{n \to \infty} \abs{\frac{a_{n+1}}{a_n}} = 1\), then the test is \underline{inconclusive}*.
\end{enumerate}

*use another test.

\subsection{ROOT TEST}

\begin{enumerate}[label=\protect\circled{\roman*}]
  \item If \( \lim_{n \to \infty}\sqrt[n]{\abs{a_n}} = L < 1\), then \( \sum_{n=1}^{\infty} a_n \) is \underline{absolutely convergent}.
  \item If \( \lim_{n \to \infty}\sqrt[n]{\abs{a_n}} = L > 1\) or \( \lim_{n \to \infty}\sqrt[n]{\abs{a_n}} = \infty\), then \( \sum_{n=1}^{\infty} a_n \) is \underline{divergent}.
  \item If \( \lim_{n \to \infty}\sqrt[n]{\abs{a_n}} = 1\), then the test is \underline{inconclusive}*.
\end{enumerate}

*use another test.

\subsection{POWER SERIES}

A power series is a series of the form

\[ \sum_{n=0}^{\infty} c_n x^n = c_0 + c_1 x + c_2 x^2 + c_3 x^3 + ...\]

where \(x\) is a variable and the \( c_n \)'s are the coefficients of the series.

A power series may \underline{converge} for some values of \(x\) and \underline{diverge} for other values of \(x\).
The sum of the series is a function

\[ f(x) = c_0 + c_1 x + c_2 x^2 + ... + c_n x^2 + ...\]

whose domain is the set of all \(x\) for which the series \underline{converges}.\bigskip

In general, a series of the form

\[ \sum_{n=0}^{\infty} c_n (x-a)^n = c_0 + c_1 (x-a) + c_2 (x-a)^2 + ...\]

is called a power series in \((x-a)\) or a power series centered at \(a\) or a power series
about \(a\).

\subsubsection{THEOREM}

For a given power series \( \sum_{n=0}^{\infty} c_n (x-a)^n \) there are only three possibilities:

\begin{enumerate}[label=\protect\circled{\roman*}]
  \item The series \underline{converges} only when \(x=a\)
  \item The series \underline{converges} \( \forall x \)
  \item There is a positive number \(R\) such that the series \underline{converges} if \( \abs{x-a} < R \) and \underline{diverges} if \( \abs{x-a} > R \).
\end{enumerate}

In general, the \underline{Ratio Test} (or sometimes the Root Test) should be used to determine the radius of convergence \( R \). The \underline{Root} and \underline{Ratio} Tests \textcolor{red}{\textit{always}} fail if \(x\) is an endpoint
of the interval of convergence, so the endpoints must be checked with some other test.

We can represent certain types of functions as sums of power series by manipulating geometric series or by differentiating such a series.
Expressing a known function as a sum of infinitely many terms is useful for integrating functions that don't have elementary
antiderivatives, for solving different equations, and approximating functions by polynomials.

Recall that:

\[ \frac{1}{1-x} = 1 + x + x^2 + x^3 + \dots = \sum_{n=0}^{\inf} x^n \qquad \abs{x} < 1 \]



\subsubsection{DIFFERENTIATION AND INTEGRATION OF POWER SERIES}

If the power series \( C_n(x-a)^n \) has radius of convergence \( R > 0 \) then the function \( f \) defined by

\[ f(x) = c_0 + c_1 (x-a) + c_2 (x-a)^2 + ... = \sum_{n=0}^{\infty} c_n (x-a)^n \]

is differentiable (and therefore continuous) on the interval \((a-R, a+R)\) and

\begin{enumerate}[label=\protect\circled{\roman*}]
  \item \( \frac{d}{dx} [\sum_{n=0}^{\infty} c_n (x-a)^n] = \sum_{n=0}^{\infty} \frac{d}{dx} [c_n (x-a)^n]\)
  \item \( \int [\sum_{n=0}^{\infty} c_n (x-a)^n] dx = \sum_{n=0}^{\infty} \int c_n (x-a)^n dx\)
\end{enumerate}

The radii of convergence of the power series in i and ii are both \( R \)

\subsection{TAYLOR SERIES}

If \( f \) has a power series representation (expansion) at \( a \), that is, if

\[ f(x) = \sum_{n=0}^{\infty} c_n (x-a)^n, \abs{x-a} < R\]

then its coefficients are given by the formula

\[ c_n = \frac{f^{(n)} (a)}{n!}\]

Substituting \( c_n \) back to the series gives

\[ f(x) = \sum_{n=0}^{\infty} \frac{f^{(n)} (a)}{n!} (x-a)^n\]

The above series is called the Taylor series of the function \( f \) at \( a \).

\subsection{MACLAURIN SERIES}

The special case \( a = 0\) of the Taylor series, such series becomes

\[ f(x) = \sum_{n=0}^{\infty} \frac{f^{(n)}(0)}{n!} x^n = f(0) + \frac{f'(0)}{1!} x + \frac{f''(0)}{2!} x^2 + \dots\]

This case arises frequently enough and is called the Maclaurin series.

\subsubsection{BINOMIAL SERIES}

If \(k \in \mathbb{R} \) and \( \abs{x} < 1 \), then

\[(1+x)^k = \sum_{n=0}^{\infty} (\nobarfrac{k}{n}) x^n = 1 + kx + \frac{k(k-1)}{2!} x^2 + \frac{k(k-1)(k-2)}{3!} x^3 + \dots \]

\subsubsection{MACLAURIN SERIES AND THEIR RADII OF CONVERGENCE}

\[ \frac{1}{1-x} = \sum_{n=0}^{\infty} x^n = 1 + x + x^2 + x^3 + \dots \qquad R = 1\] 
\[ e^x = \sum_{n=0}^{\infty} \frac{x^n}{n!} = 1 + \frac{x}{1!} + \frac{x^2}{2!} + \frac{x^3}{3!} + \dots \qquad R = \infty\]
\[ sinx = \sum_{n=0}^{\infty} (-1)^n \frac{x^{2n+1}}{(2n+1)!} = x - \frac{x^3}{3!} + \frac{x^5}{5!} - \frac{x^7}{7!} + \dots \qquad R = \infty \]
\[ cosx = \sum_{n=0}^{\infty} (-1)^n \frac{x^{2n}}{(2n)!} = 1 - \frac{x^2}{2!} + \frac{x^4}{4!} - \frac{x^6}{6!} + \dots \qquad R = \infty \]
\[ tan^{-1}x = \sum_{n=0}^{\infty} (-1)^n \frac{x^{2n+1}}{2n+1} = x - \frac{x^3}{3} + \frac{x^5}{5} - \frac{x^7}{7} + \dots \qquad R = 1 \]
\[ ln(1 + x) = \sum_{n=0}^{\infty} (-1)^{n-1} \frac{x^n}{n} = x - \frac{x^2}{2} + \frac{x^3}{3} - \frac{x^4}{4} + \dots \qquad R = 1 \]
\[ (1 + x)^k = \sum_{n=0}^{\infty} (\nobarfrac{k}{n}) x^n = 1 + kx + \frac{k(k - 1)}{2!} x^2 + \dots \qquad R = 1 \]




\end{document}